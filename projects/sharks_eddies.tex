\clearpage
\chapter{}
\section{Sharks in eddies}

\addcontentsline{toc}{subsection}{Summary and status}
\subsection*{Summary and status}
\hrule
Here is a summary of some results from this project.


%%%%%%%%%%%%%%%%%%%%%%%%%%%%%%%%%%%%%%%%%%%%%%
%%%%%%%%%%%%%%%%%%%%%%%%%%%%%%%%%%%%%%%%%%%%%%
\addcontentsline{toc}{subsection}{To-do list}
\subsection*{To-do list}
\hrule

\begin{enumerate}
\item Item 1
\end{enumerate}


%%%%%%%%%%%%%%%%%%%%%%%%%%%%%%%%%%%%%%%%%%%%%%
%%%%%%%%%%%%%%%%%%%%%%%%%%%%%%%%%%%%%%%%%%%%%%
\addcontentsline{toc}{subsection}{Research log}
\subsection*{Research log}
\hrule
%%%%%%%%%%%%%%%%%%
\vspace{10pt}

\begin{large} \textbf{01 Jan 2019} \end{large} \hrulefill
\\
%%%%%%%%%%%%%%%%%%
Details for the overall workflow of the collocation routine.

\begin{minted}{r}
## Intro
Goal is to collocate Argos-based positions from tagged marine animals to mesoscale eddies from the AVISO eddy atlas.

## Prepare the movement data

## a function that takes Argos data (from eTUFF), some simulation parameters (opt) and optional PSAT data (i.e. double-tagged fish; from eTUFF)
## output should be of class move
build_move()

## get (if needed), read (req'd) and prepare (if needed such as spat/temp subset) the AVISO eddies atlas.
eddies <- get_eddies()

## do all the eddy prep (if needed) like meander filters, calc net zonal displacement, etc
prep_eddies()

## Collocate and selection
Here we collocate the movement data to the eddies and calculate some basic stats. Also quantify eddy "selection" and/or R_pred

## take data of class move and the prepped eddies (maybe custom class also?) and produce stats
collocate_eddies()

## some way of quantifying selection or orientation to/from eddies. maybe R_pred or step selection function
choose_eddies()

## build eddy-centric composites (HYCOM?) and their anomalies (WOA?). probably should be based on region and/or eddy origin
composite_eddies()

## Plotting
Build eddy composites, static collocation figures, and animations.

## plot eddy-centric 2d and 3d composites (separate internal functions?). 3d primarily temperature. 2d could be temp, chl, etc. probably should be based on region and/or eddy origin
plot_composite()

## figs similar to blues3d Fig 1: static fig (one per indiv or taxa per region?). includes eddy collocation circles, eddy-centric histograms, and a map
plot_eddies()

## 
animate_eddies()


\end{minted}









%%%%%%%%%%%%%%%%%%%%%%%%%%%%%%%%%%%%%%%%%%%%%%
%%%%%%%%%%%%%%%%%%%%%%%%%%%%%%%%%%%%%%%%%%%%%%

\bibliographystyle{agsm}
\bibliography{Mendeley}